\documentclass[]{article}
\usepackage{graphicx}
\usepackage{subfig}
\usepackage{listings}
%opening
\title{Trabajo Pr\'actico 1 - SO - ITBA}
\author{Lpinilla}
\date{30 Octubre 2018}

\begin{document}
	
\maketitle

\subsection*{Especificaciones t\'ecnicas}
Instrucciones de compilaci\'on y ejecuci\'on y donde se corri\'o
El Trabajo Pr\'actico fue siempre compilado y ejecutado con la ayuda de un archivo makefile inclu\'ido en el repositorio como parte del trabajo pr\'actico.
	
\section*{Decisiones durante del desarrollo}
	contenidos...
	
\section*{Limitaciones}

	contenidos..

\section*{Problemas encontrados en el desarrollo y posibles soluciones}

\subsection{Uso compartido de punteros entre procesos}
Un problema con el que nos encontramos fue el uso compartido de punteros entre procesos. Inicialmente, por cuestiones de eficiencia espacial y para simplificar otros c\'alculos, cre\'iamos que la mejor manera de manejarse con los strings de formato hash nombre archivo era guardando un puntero char * que contuviera el string.\\

Esto causaba segmentation faults ya que el proceso Vision estaba accediendo a un puntero que correspond\'ia a una zona de memoria fuera de la que le correspond\'ia.\\

La soluci\'on que encontramos para este problema fue de abandonar la idea de compartir punteros y simplemente hacer una "copia profunda" del string.

\subsection{Uso compartido de punteros en zona compartida}
Otro problema que tuvimos, similar al anteriormente mencionado fue que quer\'iamos tener un puntero al pr\'oximo elemento a agregar para facilitar la escritura y lectura. Pronto nos dimos cuenta que cuando nunca \'ibamos a poder crear este puntero ya que cada proceso accede a la memoria compartida virtualmente, con lo que sus direcciones a la memoria compartida son distintas.\\

La soluci\'on que encontramos para esto fue utilizar un valor de tipo size\_t como offset de la memoria compartida, permiti\'endonos la facilidad de acceder al \'ultimo elemento sin los problemas mencionados.

\section*{Citas a c\'odigos externos usados}

Algunos de los archivos extra\'idos de internet fueron modificados para adaptarlos al uso que se les requer\'ia

\begin{center}
	\begin{table} [!h]
	\begin{tabular}{ |c|c| }
		\hline
		Archivo & Link \\
		\hline
		Queue.c & https://codereview.stackexchange.com/questions/141238/implementing-a-generic-queue-in-c \\
		\hline
		QueueTest.c &  https://codereview.stackexchange.com/questions/141238/implementing-a-generic-queue-in-c \\
		\hline
		Tasteful.c & https://github.com/lpinilla/Tasteful \\
		\hline
	\end{tabular}

	\end{table}
\end{center}
	

\end{document}